\subsection{Level 1.1: コンピュータと人間の違いを述べよ}
\subsubsection{課題説明}
コンピュータが人間より得意とするモノ、その反対に人間より不得手のモノ、両者について2つ以上の視点(立場や観点など)を示し、考察する。

\subsubsection{考察}
\begin{itemize}
 \item 視点1: 計算\\
	コンピュータならば,短時間で複雑な計算が可能であり,私たちも普段複雑な計算をする際にはコンピュータまたは計算機を用いていることから得意であることが明らかである.\\
	だが人間は,簡単な計算ならば,計算しなくても暗算で求めることができるため,複雑な計算はコンピュータ,容易な計算は人間が得意である. 

 \item 視点2: 認識\\
	人間は視覚,聴覚,感覚,嗅覚,味覚の五感があり,それを組み合わせて用いることで対象を正確に認識することができ,初めて見るものでも記憶にあるものと比較しそれがなんであるか推測することが可能である.\\
	コンピュータは,様々なセンサなどで検出したデジタルデータを元に対象を認識しているため,センサで検出したデータが不足している場合,正しく認識することができない.\\
よって認識については人間のほうが得意であるが,  
Siriなど音声を認識し,人間を会話できるほどの機能をもつアプリケーションも存在するので,今後は人間より認識できるようになる可能性がある.  

 \item 視点3: 学習・探索\\
	コンピュータは,単純作業を繰り返し行う場合に短時間で行うことが可能であるため,
パラメータを細かく変更して最適値を求めたり,それぞれのデータに重みをつけて分別またはクラスタリングしたりすることが得意である.\\
	人間は,「考えること」が可能である.
認識でも話したが,対象を過去に五感で感じ,学習したものと比較し,推測することができたり,未知の物事について研究し,それがなんであるかを考え学習することができる.

 \item 視点4: 記憶\\
	コンピュータは,記憶を劣化せずに保持することが可能なため,検索が得意である.
また,学習・探索で話したが,単純作業が得意なため,ソートも人間より早く正確に行うことができる.\\
	人間は,記憶を劣化させてしまうため,検索・ソートについては不得意である.

 \item 視点5: ゲーム\\
	コンピュータは,ルールが単純であるゲームであれば,現状を評価し,最適な手段を探索することができる.
現在ではチェスはコンピュータに勝つことができなくなっている.\\
	人間は,ルールを理解し深めることでどんなゲームで最適な手段を探索でき,最適な手段が複数あった場合に状況にあった手段を選択することができる.

\end{itemize}

