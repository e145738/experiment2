\subsection{課題説明}

最急降下法が苦手とする状況についてその理由を解説し,
検討した改善方法について解説する.

\subsubsection{原因}

「山(谷)」の数は一つにも関わらず最も勾配の高い方向に移動するため,一直線に谷には向かわずジグザグ移動してしまい,効率的ではない.

\subsubsection{改善方法}

・複数の初期値から探索を行いより短い探索で解にたどり着ける初期値を探す.\\
最適解の近傍に初期値を設定することができれば,余計な探索を減らすことができると考えた.
最適解のx軸方向の直線上に近い場所に初期値を設定することができれば,一直線に谷に向かうことも可能となる.\\\\


・また収束速度を安定させる手法として,前回得られた勾配ベクトルと今回得られた勾配ベクトルの符号の変化によって刻み幅の調整をする方法がある.符号が反転したときに刻み幅を小さくし,符号が等しいときに刻み幅を大きくすることで安定して収束する.





