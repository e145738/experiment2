\subsection{Level3.2: パラメータと収束能力の関連性について}
\subsubsection{関係性を確認するためのアプローチ}
HIDDENを固定してのETAの変更、ALPHAの変更を行っていき、値の変動を見て法則を確認する。

\subsubsection{結果}
ETAの変更による値の変化\\
HIDDEN : 30\\
ETA        : 0.5\\
ALPHA   : 0.72\\

指定シード値10パターンによる平均値 : 236\\

HIDDEN : 30\\
ETA        : 0.6\\
ALPHA   : 0.72\\

指定シード値10パターンによる平均値 : 216.9\\

HIDDEN : 30\\
ETA        : 0.7\\
ALPHA   : 0.72\\

指定シード値10パターンによる平均値 : 187\\


~中略~\\

HIDDEN : 30\\
ETA        : 1.3\\
ALPHA   : 0.72\\

指定シード値10パターンによる平均値 : 108\\


HIDDEN : 30\\
ETA        : 1.4\\
ALPHA   : 0.72\\

指定シード値10パターンによる平均値 : 129.5\\


ALPHAの変更\\
HIDDEN : 30\\
ETA        : 1.3\\
ALPHA   : 0.73\\

指定シード値10パターンによる平均値 : 109.4\\


HIDDEN : 30\\
ETA        : 1.3\\
ALPHA   : 0.71\\

指定シード値10パターンによる平均値 : 113.3\\


HIDDENとETAの関係について\\
HIDDEN : 31\\
ETA        : 1.3\\
ALPHA   : 0.72\\

指定シード値10パターンによる平均値 : 93.1\\


HIDDEN : 31\\
ETA        : 1.4\\
ALPHA   : 0.72\\

指定シード値10パターンによる平均値 : 105.1\\


HIDDEN : 31\\
ETA        : 1.2\\
ALPHA   : 0.72\\

指定シード値10パターンによる平均値 : 92\\


HIDDEN : 31\\
ETA        : 1.1\\
ALPHA   : 0.72\\

指定シード値10パターンによる平均値 : 88.1\\


HIDDEN : 31\\
ETA        : 1.0\\
ALPHA   : 0.72\\

指定シード値10パターンによる平均値 : 99.1\\

\subsubsection{考察}
上記実行結果を総合して考えた結果\\
HIDDENが大きくなると、早く収束する点でのETAの値はある一定まで小さくなり、HIDDENが小さくなると、早く収束する点でのETAの値は一定まで大きくなる、という関係があるように思えた。




