\subsection{Level3.4: 認識率を高める工夫}
\subsubsection{対象とする問題点}
入力されたデータが想定していた入力と比べてサイズが異なったり、位置がずれている等、文字の一部が欠けている以外にも多様な要因によるデータ(情報)の劣化が考えられる。認識率を高めるにはどのような点を工夫すれば良いか?どのような方法でも構わないので、検討せよ。

\subsubsection{改善方法の提案}
認識率を高めるために改善すべき方法として,学習データの数を増やすということがあげられる.

\subsubsection{考察}
今回の実験では,学習サンプルデータがそれぞれ1種類ずつしかなかった.そのため,いろいろなパターンの文字の形を学習させる際に少々学習に手間取ったように見える.この問題を解決するための方法として学習データの増加を行う必要がある.

人の脳をモデルとしているのならば,人が行うような学習をさせればいいのだから,いろいろなデータを蓄えておいて,それを元に分析を行うことはごく自然なことである.一つの事象にとらわれることなく学習を行うために,データの増加は必要だと考えた.1つの文字に対して複数のパターンを用意し,それに基づいて学習をさせれば,与えたデータをより細かく学習させることができるだろう.

また,複数のパターンを用意して,それに共通する部分や同じような形を記憶させておけば,さらに学習パターンは広がるはずだ,これを実現させるためにはさらにコードを書かないといけなくなるので,容易にできるのはデータ数を増やすことだと考えた.